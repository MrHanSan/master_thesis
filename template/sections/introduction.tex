%!TEX root = ../report.tex
\chapter{Introduction}
\label{cha:Introduction}

Creating efficient structures for computers to interpret information opens up many new areas for information retrieval. One method is structuring the information as a graph with vertices connected by edges (G(v, e)). In such a structure each vertex (v) can hold a small piece of information, and the edge (e) represents a relation. When information is modeled in this way, it can be called a knowledge graph (KG).

Graph based models allows complex relations to be modeled. Finding relations in a graph is also computationally cheaper than other models. Using standardized models such as the Resource Description Framework (RDF), which is created specifically for graph based data, has made it possible to develop efficient storage and retrieval solutions for graph data. One such solution is the framework Jena \citep{jena2} for Java.

Using RDF projects like YAGO (Yet Another Great Ontology) \citep{yago} have created graphs containing spatial and temporal data (spatiotemporal). This makes it possible to apply reason to the data set, and derive new information from what is already there. This is a feature that typically separates a KG from an ontology \citep{KGDefYago}. In a graph, spatiotemporal data will be described using at least 2 vertices, one for a start date, and one for a place. When modeling time intervals, multiple vertices are needed.

There are few applications using RDF as an information storage system. Some of the more well known are DBPedia, YAGO, and Creative Commons. These applications often consists of a large linked data set, or in the case of Creative Commons, it is used for embedding licenses. The applications built with RDF usually don't have much utility. A common utility built on top of RDF data sets are tools used to describe social relations, though there have been developed applications such as MusicBrainz that use RDF for relations between songs, artists, albums and other tags given to entities.

Creating methods that allows for easy access to information in RDF data makes it possible to create applications with more utility. Keyword and text search in RDF data is still something being developed and improved. Using frameworks such as Jena, it is possible to build solutions for keyword search, and spatiotemporal search. In this thesis (paper?) methods for efficient keyword search on spatial and temporal data will be explored.


\section{Background and Motivation}
\label{sec:BackgroundAndMotivation}
Traditionally relationships between pieces of data have been difficult for computers to model. Regular SQL databases requires expensive joining to be able to display relationships between entities. As more data is generated, there are also more relations between the data. Exploring these relations can be done by using RDF or other graphs, but for a regular person this can be difficult. By proving that fast and accurate keyword search of spatiotemporal RDF graphs is possible new utilities for exploration can be built.


\section{Goals and Research Questions}
\label{sec:Goals and Research Questions}
\begin{description}
    \item[Goal] {\em With this thesis the goal is to determine if and how spatiotemporal keyword searching in large scale RDF graphs is possible to do in fast and accurate.}
\end{description}
% Why this goal
Accomplishing this goal proves that RDF graphs can be used as a tool for structuring data related to real world places. There is a lot of data that can be placed at one or more real locations, either directly, or indirectly. By using RDF graphs it is possible to model how different places are connected through some pice of data, and how different pieces data can be related to real places.

\begin{description}
    \item[Research question 1] {\em How can spatiotemporal data be integrated into exiting keyword query methods for RDF data.}
\end{description}

\begin{description}
    \item[Research question 2] {\em What methods can be used to create more effective queries on RDF data.}
\end{description}

\begin{description}
    \item[Research question 3] {\em How do spatial and temporal RDF query methods differ from from other query methods.}
\end{description}

\section{Research Method}
\label{sec:researchMethod}
This thesis will build on previous keyword search approaches for RDF graphs, and determine what methods can be expanded to incorporate spatiotemporal queries. A method for spatiotemporal search will be created, and methods for improvement will be tested, with the goal of finding what aspects of the search method have most effect for speed and accuracy.

\section{Thesis Structure}
\label{sec:thesisStructure}

\glsresetall