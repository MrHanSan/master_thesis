%!TEX root = ../report.tex
\chapter{Introduction}
\label{cha:Introduction}

Creating efficient structures for computers to interpret information opens up many new areas for information retrieval. One method is structuring the information as a graph with vertices connected by edges (G(v, e)). In such a structure each vertex (v) can hold a small piece of information, and the edge (e) represents a relation. When information is modeled in this way, it can be called a knowledge graph (KG).

Graph based models allows complex relations to be modeled. Finding relations in a graph is also computationally cheaper than other models. Using standardized models such as the Resource Description Framework (RDF), which is created specifically for graph based data, has made it possible to develop efficient storage and retrieval solutions for graph data. One such solution is the framework Jena \cite{jena2} for Java.

Using RDF projects like YAGO (Yet Another Great Ontology) \cite{yago} have created graphs containing spatial and temporal data (spatiotemporal). This makes it possible to apply reason to the dataset, and derive new information from what is already there. This is a feature that typically separates a KG from an ontology \cite{KGDefYago}. In a graph, spatiotemporal data will be described using at least 2 vertices, one for a start date, and one for a place. When modeling time intervals, multiple vertices are needed.

There are few applications using RDF as an information storage system. Some of the more well known are DBPedia, YAGO, and Creative Commons. These applications often consists of a large linked dataset, or in the case of Creative Commons, it is used for embedding licenses. The applications built with RDF usually don't have much utility. A common utility built on top of RDF data sets are tools used to describe social relations, though there have been developed applications such as MusicBrainz that use RDF for relations between songs, artists, albums and other tags given to entities.

Creating methods that allows for easy access to information in RDF data makes it possible to create applications with more utility. Keyword and text search in RDF data is still something being developed and improved. Using frameworks such as Jena, it is possible to build solutions for keyword search, and spatiotemporal search. In this thesis (paper?) methods for efficient keyword search on spatial and temporal data will be explored.


\section{Background and Motivation}
\label{sec:BackgroundAndMotivation}


\section{Goals and Research Questions}
\label{sec:Goals and Research Questions}
Research questions?\\
RQ1: How can spatio-temporal data be integrated into exiting keyword query methods for RDF data.\\
RQ2: What methods can be used to create more effective queries on RDF data.\\
RQ3: How do spatial and temporal RDF query methods differ from from other query methods.\\

\section{Research Method}
\label{sec:researchMethod}

What methodology will you apply to address the goals: theoretic/analytic, model/abstraction or design/experiment? 
This section will describe the research methodology applied and the reason for this choice of research methodology.  

\section{Contributions}
\label{sec:IntroContributions}

This section just provides a brief summary of the main contributions of the work. 
The main description of the contributions will come in Section~\ref{sec:Contributions}, after the results are presented. 
(Hence Section~\ref{sec:IntroContributions} can also be left out, leaving the discussion completely to Section~\ref{sec:Contributions}.)

The format of this section will generally be as follows:
{\em
Donec non turpis nec neque egestas faucibus nec id neque. Etiam consectetur, odio vitae gravida tempus, diam velit sagittis turpis, a molestie ligula tellus at nunc. Nam convallis consequat vestibulum. Proin dolor neque, dapibus a pellentesque a, commodo a nibh.}

\begin{enumerate}
\item {\em Lorem ipsum dolor sit amet, consectetur adipiscing elit.}
\item {\em Lorem ipsum dolor sit amet, consectetur adipiscing elit.}
\item {\em Lorem ipsum dolor sit amet, consectetur adipiscing elit.}
\end{enumerate}

\noindent
where the items on the list briefly describe the key contributions.

\section{Thesis Structure}
\label{sec:thesisStructure}

This section provides the reader with an overview of what is coming in the next chapters. 
You want to say more than what is explicit in the chapter name, if possible, but still keep the description short and to the point. 

\glsresetall