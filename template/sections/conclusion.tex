%!TEX root = ../report.tex
\chapter{Conclusion and Future Work}
\label{cha:Conclusion}

\section{Contributions}
\label{sec:Contributions}
This thesis had the goal to research methods for keyword searches on large spatiotemporal RDF graphs. This was broken down into three research questions.

\begin{description}
    \item[Research question 1] {\em How can spatiotemporal data be integrated into exiting keyword query methods for RDF data.}
\end{description}
In the section \ref{cha:related_work} previous methods used for keyword searches on RDF graphs was investigated. Using methods from other research spatiotemporal keyword search have been implemented. The effectiveness of the search can be further improved, and implementing methods such as natural language processing for categorizing the keywords could further increase speed, without being detrimental to accuracy.

\begin{description}
    \item[Research question 2] {\em What methods can be used to achieve greater speed and accuracy for searches on RDF data.}
\end{description}

This thesis looked at the parameters of a search that could increased speed, and decreased accuracy. When increasing speed, the main goal should be to reduce the amount of root vertices used as starting points for a search. Reducing the edges followed will also have a great effect on the speed. To be able to do this, processing the query, so that a structure can be created from the keywords is a possibility.

\begin{description}
    \item[Research question 3] {\em How do spatial and temporal RDF query methods differ from from other query methods.}
\end{description}

Using a BFS for traversing the graph, a spatiotemporal search does not need to differ from spatial or temporal search. When selecting the roots used as starting points for the search, a spatiotemporal search needs to find roots that fall into both the spatial dimensions of the search, and the temporal. This makes the total set of roots used for a spatiotemporal search less than if the search was done just on one dimension.

\section{Future Work}
\label{sec:futureWork}
Combining structured queries natural language processing with graph traversal can yield grat bennigfits for ssearching graphs. 

Indexing and traversal form keywords and not just nodes is a possibility for increasing the effectivness of searches. Adding toi this, finding shortest paths in the graph is another method for fiding results.
