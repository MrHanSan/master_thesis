%!TEX root = ../report.tex
\chapter{Architecture}
\label{cha:architecture}
In this chapter the implementation of search and traversal is described, along with a method for ranking results. When building a system for keyword search, the first implementations created for this thesis was based on previously created systems. This early method was based on \cite{Shi:2016:TRS:2882903.2882941} and the method was also used as the base for adding a temporal dimension to the search.

\section{Search}
Each search will return a subgraph, based in a root. This forms a tree structure that can be described as follows:
\begin{description}
    \item[Result tree] {\em Result r for a given query with tokens $T_q$ on an RDF graph G$\langle$V, E$\rangle$, where the result forms a minimum spanning tree $M_t\langle$V', E'$\rangle$, V' contains a set of tokens $T_t$, $M_t$ is rooted in a place vertex, so that V' $\subseteq$ V, E' $\subseteq$ E, and $T_t \subseteq T_q$}
\end{description}
All results from a query is given as a minimum spanning tree. This tree is call a {\em Result Tree}. A query can have multiple result trees, where each result tree is rooted in a unique place vertex, and each tree contains at least one query word.

A root vertex can be described as:
\begin{description}
    \item[Root] {\em Starting point of a BFS traversal, and the vertex all other vertices are connect to in a result tree.}
\end{description}

A spatiotemporal root is like any other root, but with an addition:
\begin{description}
    \item[Spatiotemporal root\label{rootST}] {\em Any root R where the spatial input $I_s$ and temporal input $I_t$ overlaps, so that $R \in I_s \cap I_t$.}
\end{description}

One basic method of finding a match for keywords is using a breadth first search (BFS) \cite{blinks, Shi:2016:TRS:2882903.2882941}. For each keyword in the query the algorithm will find all vertices that contain the keyword. From that set of vertices, the BFS finds the first vertex that can connect all the vertices in the set. When this vertex is found, the whole subgraph is returned as a result.

Using BFS starting on keyword vertices works for keyword search, but not necessarily when searching for spatial or temporal data. When adding spatial or temporal vertices to the search, these can be used as a root and a starting point for traversal. When a search initiates, a place or time vertex existing in the graph needs to be selected. For spatial search, all places connected with the ``isLocatedIn'' predicate will be used as roots when searching. This ensures that the roots are geographically located within the selected place. For temporal searches all vertices containing the queried time or within the time span is used as roots. From the root vertices the search will traverse the graph looking for vertices matching the keywords or query terms, and when all terms are found, that tree is the best match for the start root. After all roots are searched, the trees are given a score based on how well they fit the query.

The first part of the search is to find a set of root vertices used as the starting point for traversal. This is done using SPARQL to query specific data in the graph matching the input place or the input time. The neighbors of the roots can be connected by any predicate in the first developed algorithm, but as explain in section \ref{pruning} selecting specific predicates can greatly increase speed by limiting the number of vertices visited. The BFS algorithm described in algorithm \ref{fullRT} is used on each of the possible roots, stopping when the distance from the root exceeds a threshold, or the distance from the root exceeds the currently shallowest tree that contains all query terms. The reason for limiting the distance from root is to ensure at least a partial hit within a reasonable time. Limiting the distance from root to the current shallowest tree is done because a tree with a greater distance from root will be scored lower.

When traversing the graph an object is created for each vertex visited. This object is used to keep track of the pseudo hierarchy in the graph created during traversal. All objects contain information on the distance from the root, matched query terms, and relation to parent, if any. In addition, root objects contain a list of all vertices in the tree with at least one query term match. If a child vertex is found within multiple trees, a new object will be created representing the vertex for each tree. This creates some objects that are nearly identical, but with different relations and possible different distance.

When a vertex object is created, a document with terms is created for that vertex. This document is used to calculate score, and to match a vertex with the query terms. A document is created from the last part of a Yago URI, after the last slash. Each URI is further split on each underscore, creating a list of terms.

After traversing the main graph, we have found all trees containing at least one query term. These trees may contain many vertices which hit the same terms. Before ranking the trees, the minimum spanning tree needs to be found. The minimum spanning tree is found using a greedy algorithm that iterates through all vertices in the tree, then keeping the vertices containing the most terms, at the lowest distance. The minimum tree will contain as many terms as possible, each term will only be found in one vertex and the tree will be as shallow as possible.

In the graph traversal implementation, a threshold for maximum distance from root is set. This threshold is updated if all query terms are found within the tree. The traversal will still check the rest of the queued items, even if all query terms are found. This is done because some might be a better match than what is already added. The vertices found during traversal is added to a list of vertices, and each vertex have a link to the parent, where the parent is the current vertex and the children are the vertices discoverable from the parent.
\begin{algorithm}[H]
    \caption{GetFullResultTree($V_r$, t, $T_q$)}
    \label{fullRT}
    \SetAlgoLined
    \KwResult{Set containing all vertices with at least one keyword within threshold}
    Array $\mathbf{Q}$ Add($V_r$)\; Threshold t\; Set $R_t$\;
    \While{$\mathbf{Q} \neq \emptyset$}{
        $v$ = GetFirstElement($\mathbf{Q}$)\;
        \If{GetDistance($v$) > t}{
            continue\;
        }
        \ForEach{Term T $\in T_q$}{
            \If{T $\in$ $v$}{
                $V_r$ AddMatchChild($v$)\;
            }
        }
        \If{$T_q$ = $V_r$ AllMatchedTerms()}{
            t = GetDistance($v$)\;
        }
        $n$ = GetConnectedVertices($v$)\;
        \ForEach{Child c $\in$ $n$}{
            c AddParentNode($v$)\;
        }
        $\mathbf{Q}$ Add($n$)\;
        $R_t$ Add($v$)\;
    }
    \Return{$R_t$}
\end{algorithm}

\section{Ranking}
All result trees are given a score based on how well they fit the query. This score is called accuracy and is made up of two parts. The first part is how well a vertex fits the query terms, and the other is how far it is removed from the root. In algorithm \ref{minResultTree} a minimum tree is created by discarding all nodes except those with the most query term hits, at the shortest distance from root. If multiple vertices are found at the same distance and with the same number of hits, the first to be added is used in the tree. By using the connection to the parent vertex, a minimum tree is reconstructed from the best vertices.
\begin{description}
    \item[Accuracy:] {\em Score given to a minimum spanning tree based on a result tree. Score is based on the number of vertices  in the minimum tree, vertex distance $D_v$ from root, query q, and number of query words hit h so that $\mathcal{F}_h = (h_i/\left\lvert q \right\rvert : i \in \mathcal{I})$ and $ \frac{\sum \mathcal{F}_h}{n*(D_v+1)}$}
\end{description}
To find the accuracy, a set containing the best vertices is first extracted from the result tree. To extract these vertices, the algorithm \ref{minResultTree} is used.
\begin{algorithm}
    \caption{FindMinimumTree($R_t$)}
    \label{minResultTree}
    \SetAlgoLined
    \KwResult{A set of vertices best fitting the query input}
    ResultTree $R_t$\; MinimumTree $M_t$ Add($R_t[0]$)\;
    \ForEach{Vertex r $\in R_t$}{
        \If{r hits = $\emptyset$}{
            continue\;
        }
        \ForEach{Vertex m $\in M_t$}{
            \If{r hits = m hits $\land$ r distance > m distance}{
                Break\;
            }
            \If{r hits = m hits $\land$ r distance = m distance}{
                Continue\;
            }
            \If{r hits $\neq$ m hits $\vee$ r distance < m distance}{
                m RemoveCommonWords(r hits)\;
                $M_t$ Add(r)\;
            }
            \If{m hits = $\emptyset$}{
                $M_t$ Remove(m)\;
            }
        }
    }
    \Return{$M_t$}
\end{algorithm}

\section{Pruning}
\label{pruning}
When using the BFS search method, all possible spatial vertices within the input place will be explored. This is expensive and many of the vertices will be irrelevant. Pruning the potential place vertices will reduce the number of subgraphs traversed and will in turn reduce the overall time used to find results for the query.

\subsection{Predicate selection}
When traversing the graph, a lot of unnecessary predicates are followed, resulting in many extra vertices added to the search. Specifying a set of predicates that contain the relevant information can greatly increase the speed of the algorithm and keep the memory requirements a lot lower. When selecting predicates for traversal the information retrieved must be related to the input. Because of this, all predicates that may contain spatial or temporal data should be kept, based on the type of query.

When using the entirety of the Yago data set, most vertices are highly connected. Many of the links in the graph are from predicates such as ``linksTo'' or ``redirectedFrom''. These predicates create a highly connected graph and ensures a hit within a few vertices of the start. The same predicates will also often add the same vertices multiple times, create circular graphs, and take up unnecessary CPU power and memory.

When pruning predicates there are two methods that are possible to implement. The first method will remove the predicates that contain little or no new information, such as ``linksTo'' or ``redirectedFrom'' mentioned above. This will keep the graph connected, and keeps the predicates containing more useful information. Another method of pruning is to create a predefined list of predicates to be followed. This can drastically reduce the connections in the graph, but the results will only contain information relevant to the query. When preselecting predicates there is a much greater chance of not finding a match for a query. In addition, a lot of metadata could be lost if the metadata predicates are not added to the list of predicates to be explored.

\subsection{Place hierarchy}
In YAGO places form a hierarchy by using the ``isLocatedIn'' predicate. This predicate can be used to find places that is as detailed or more detailed than the input place selected by moving down the hierarchy. To find the best possible matches for a spatial query, only places located within the selected input, or the input itself should be used as roots. This means that places of similar or smaller expanse should be allowed to be queried, e.g. a query of Boston can return the entirety of Boston, or within Boston. Massachusetts has multiple predicates linking it to Boston but should not be queried because the information returned would be less detailed than the input. 

\glsresetall