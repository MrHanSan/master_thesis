%!TEX root = ../report.tex

\begin{otherlanguage}{norsk}

\section*{Sammendrag}
Denne oppgaven tar for seg metoder for nøkkelord søk i RDF data, med mål om å finne hvilke aspekter som er nødvendig for å kunne gjennomføre søk med høy nøyaktighet og hastighet. For å oppnå dette vil tidligere metoder bli utforsket og utvidet til å inkludere søk både i tid og sted.

Oppgaven bruker bredde først søk på forskjellige data, og med forskjellige vilkår for å innhente data. Dataen er delt inn i tid, sted, og tid og steds data, hentet fra YAGO datasettet. Med tre forskjellige vilkår for traversering av grafen vil data bli innhentet for å finne aspekter som kreves for hurtige og nøyaktige søk.

Resultatene viser at både nøyaktighet og hurtighet avhenger av hvor mange noder blir besøkt under traversering. Ved å minske antall noder og kanter som blir fulgt gjennom fjerning og filtrering vil hastigheten på søk øke.

Oppgaven viser at eksisterende metoder for søk i RDF grafer kan utvides til å inneholde søk i tids- og stedsdata. Ved å bruke mer sofistikerte metoder kan nøyaktighet og hurtighet videre forbedres, og slike metoder kan implementeres på eksisterende grafer.

\end{otherlanguage}
