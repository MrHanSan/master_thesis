%===================================== CHAP 2 =================================

\chapter{Spatio-temporal data structures}

\subsection{Spatial indexing}
Most spatial indexing is done using R-trees. R-trees are based on a B-tree, but is optimized for indexing spatial dimensions. R-trees are able to effectively index spatial data by 

\subsection{Temporal indexing}
In the dataset used in this thesis objects do not move, or move rarely. This makes indexing time unnecessary and it can be treated as a fact in the object with a start, and possible end time. This allows time to be retrieved as a fact when traversing the graph the same way as other facts.

\cleardoublepage