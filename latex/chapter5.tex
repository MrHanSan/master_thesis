%===================================== CHAP 5 =================================

\chapter{Methodology}

\section{Datasets}
\subsection{YagoFacts}
Data set containing all facts in Yago that hold between instances. This data set is the largest that is used, and holds all information.

\subsection{GeoNames}
All facts in Yago containing geographical coordinates and names. This data set contains all the place information and is used to index places and is used for the spatial indexing and used when finding a place that fits with the query or used as a starting place for spatial queries. 

\subsection{DateFacts}
All facts in Yago containing dates.

\includegraphics[width=\textwidth]{example-image-c}

\section{Evaluation}
There are multiple possible methods for evaluating the indexing and search methods. A good method for evaluating the retrieval methods is time. The faster information can be retrieved the better the retrieval method should be, assuming the information retrieved is correct. For the purpose of this theses time complexity will be the main evaluation method, along with evaluation of how well the retrieved information fits the query.

\subsection{Time complexity}  

\subsection{Space complexity}

\subsection{Information match}

\clearpage