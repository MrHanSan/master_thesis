\section*{\Huge Problem Description}
\addcontentsline{toc}{chapter}{Preface}
$\\[0.5cm]$

\noindent 
Working title (?): Top-k relevant spatiotemporal ... retrieval on knowledge graphs using keyword search % Need something more

Knowledge graphs = RDF structures (add to terms/ abbr. or something...)

parts of the task:
- Indexing
	- Spatiotemporal
	- General entities?

- Traversing
	- Compare different traditional algorithms
	- Look for any algorithms designed for spatiotemporal data
- Ranking
	- Based on traversal?
	- Based on distance between hit nodes?
	- Read more!


- Why knowledge graphs? % Use for problem description on in contract. 

% All this is just thoughts, should be cleanded up, and put somewhere (maeby?) 

Currently there is little previous work done on spatial retrieval on knowledge graphs using keywords. There have been some (ref.) attempts ...
Even less is done on spatiotemporal retrieval. Data collections such as, YAGO, holds both spatial and temporal on many different entities, making it possible to retrieve the entity data based on keyword searches.

To be able to do keyword search on RDF data the data needs to be indexed in some form, previously r-trees have been used (ref.) and graphs / sub-graphs must be traversed. Because the indexing for spatial RDF has been done previously, this thesis will not go too much in depth on that aspect. The temporal dimension has less research, so this thesis will explore this part of the problem in depth.

The task will also require some form of ranking to determine the top-k relevant results retrieved. There has also been done work on ranking RDF keyword search retrieval, so this thesis will not go in depth on the ranking. %Stop saying what not to do, it will never stop



In this project, we want to study spatio-temporal search on such graphs, aiming at developing indexes and algorithms for more efficient search.

\clearpage