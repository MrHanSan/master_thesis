%===================================== CHAP 4 =================================

\chapter{Optimizing search}

\section{Pruning} \label{pruning}
When using the BFS search method, all possible spatial vertices close to the queried place will be explored. This is expensive and many of the vertices will be irrelevant. Pruning the potential place vertices will reduce the amount of subgraphs traversed, and will in turn reduce the overall time used to find results for the query.\\

\subsection{Predicate pruning}
When traversing the graph a lot of unnecessary predicates are followed, resulting in many extra nodes added to the search. Specifying a set of predicates that contain the relevant information can greatly increase the speed of the algorithm, and also keep the memory requirements a lot lower. When selecting predicates for traversal the information expected from the search should be the top priority. Because of this, all predicates that may contain spatial or temporal data should be kept.\\

When using the entirety of the Yago data set, most nodes are highly connected. Many of the links in the graph are from predicates such as ``linksTo'' or ``redirectedFrom''. These predicates creates a highly connected graph, and ensures a hit within a few nodes of the start. The same predicates will also often add the same nodes multiple times, create circular graphs, and take up unnecessary CPU power and memory.\\

When pruning predicates there are two methods that are possible to implement. The first will remove the predicates that contain little or no new information, such as ``linksTo'' or ``redirectedFrom'' mentioned above. This will still keep the graph connected, and keeps the predicates containing more useful information.\\

An other method of pruning is to create a list of predicates to be followed. This can drastically reduce the connections in the graph, but the results will only contain information relevant to the query. When preselecting predicates there is a much greater chance of not finding a match for a query. In addition a lot of metadata could be lost, if the metadata predicates are not added to the list of predicate to be explored.\\

\subsection{Unqualified place pruning}
For any spatial search, a lot of places will be found. The graph also contains information on the relationship between places. To find the best possible matches for a spatial query, only places of a higher resolution should be chosen. This means that places of similar or smaller expanse should be allows to be queried, e.g. a query of Boston can return the entirety of Boston, close to Boston, or within Boston. Massachusetts has multiple predicates linking it to Boston, but should not be queried because the information returned would be less detailed than what is queried. \\

\subsection{Bounding}
When selecting places or times for a query, the boundaries should be set so that they encompass the entirety of the queried time and or place.\\

\clearpage