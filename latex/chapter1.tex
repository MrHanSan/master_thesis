%===================================== CHAP 1 =================================

\chapter{Introduction}
Today large collections of information is available, but for this information to be useful, efficient retrieval methods is needed. More and more information is also being generated in collections with specific uses, such as a school wiki, or an article collection for a company. All this different types of information collections need methods for retrieval.\\

Any collection of information can be called a knowledge base. Such a collection can be structured or unstructured, and can be any size. Larger collections will require a structure to be navigable, and to make it easier to search. Depending on the type, size and use of a collection, different system should be used. For a small collection not expected to grow much a set of unordered articles might be sufficient, while a collection such as Wikipedia will need indexing, linking, categories and searching for the information to be accessible.\\

In any set of information there will usually be some relation to other information. This relation can be in form of a link to another piece of information, relationships between real entities, such as the relation between a capitol and a country. Relations can also be more abstract, such as two entities being placed in the same category. Using the relationship between pieces of information it is possible to create a directed graph. Such a graph is known as a knowledge graph.\\

Knowledge graphs makes it possible to represent detailed and complex information in an ordered 

% - Information have relationships
- A user might need parts of different sources
- Finding specific information based on relationship, using part of different sources/ sets (?)
- often duplicate in articles 

%Traditional retrieval consists of large databases with indexes and web crawlers gathering information.
% - Why this is important\\


% - What is the contribution of this paper\\
This paper will propose methods of integrating time and date facts in existing search methods.\\

\section{Previous work}
There has been done some work keyword search on RDF graphs. This includes \cite{4812421} and \cite{Elbassuoni:2011:KSO:2063576.2063615}. None of these takes spatial or temporal data into consideration. The methods for keyword queries outlined in the papers are based on a combination of indexing the graph and traversing subgraphs. The indexing is similar in both papers, but retrieval, scoring and traversal methods varies. One method of retrieval and scoring that is used as a baseline is a breadth first search for traversal and a minimal subgraph for scoring. These methods are easy to implement and understand, but often inefficient, and inaccurate.\\

In paper \cite{Shi:2016:TRS:2882903.2882941} some methods for indexing, searching and ranking keyword searches on spatial RDF graphs is proposed. These methods builds upon the methods outlined in \cite{4812421, Elbassuoni:2011:KSO:2063576.2063615}. The most substantial difference is the use of R-trees to index the spatial dimension of the graph. This is done so that a subgraph containing the keywords can be retrieved given a location. This method starts by finding the closest spatial vertex and uses that as the root for the subgraph. Subgraphs rooted in a spatial node requires different ranking, and there is proposed a set of rules in the paper.\\

R-trees are designed around the spatial dimension, but some work on modifications to include a temporal dimension has been done. Some research done on spatio-temporal tree structures include \cite{Tao:2003:TOS:1315451.1315519, r-tree-spatio-temporal}. Most of the spatio-temporal trees are however created to be able to query and index moving or frequently updated objects. To include a temporal dimension some differences in the tree structures are made. Most approaches builds on the R-tree, but modifies the tree in different aspects. For the purpose of this thesis a different tree structure is not needed.\\ % look into temporal retrieval as facts in graph/ inverted index

% TPR-trees \cite{} uses some tech ... This makes them most suited to index moving objects, with as spatio-temporal data to indicate where an object is at a given time. The work done in \cite{} builds heavily on the R-tree, and makes it suited for ...

Research questions?\\
RQ1: How can spatio-temporal data be integrated into exiting keyword query methods for RDF data.\\
RQ2: What methods can be used to create more effective queries on RDF data.\\
RQ3: How do spatial and temporal RDF query methods differ from from other query methods.\\
RQ4: What methods can be used to query RDF data based on a users keyword search.\\

\clearpage