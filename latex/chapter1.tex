%===================================== CHAP 1 =================================

\chapter{Introduction}
Knowledge bases are complex collections of information. This information can be both structured or unstructured. The information in such a collection can be used to create a graph with information in small chunks representing a fact, as well as the relation between such facts. There exists multiple such graphs today most are based on Wikipedia, such as Yago\cite{yago} and DBpedia\cite{dbpedia}.\\
Knowledge graphs is structured with a subject, predicate, and object. Both the subject and object are facts, and the predicate describes the relation between two facts. This structure makes the graph a directed cyclic graph. An example of the subject predicate object can be "Nidaros Cathedral" "is located in" "Trondheim".\\
\includegraphics[width=\textwidth]{example-image-a}\\
One of the uses of knowledge graphs today is to find and display a info box in search engines. This information is a compact set of facts that tries to fit the search query. Because of the graph structure of knowledge graphs the information in the info box can be adapted to the query by choosing the predicates and related facts closest related to the query. This makes it possible to create a set of information that can give the user a quick overview of the information retrieved by the query.\\

- Why this is important\\
- What is newish in this paper\\


\section{Previous work}
There has been done some work keyword search on RDF graphs. This includes \cite{4812421} and \cite{Elbassuoni:2011:KSO:2063576.2063615}. None of these takes spatial or temporal data into consideration. The methods for keyword queries outlined in the papers are based on a combination of indexing the graph and traversing subgraphs. The indexing is similar in both papers, but retrieval, scoring and traversal methods varies. One method of retrieval and scoring that is used as a baseline is a breadth first search for traversal and a minimal subgraph for scoring. These methods are easy to implement and understand, but often inefficient, and inaccurate.\\
In paper \cite{Shi:2016:TRS:2882903.2882941} some methods for indexing, searching and ranking keyword searches on spatial RDF graphs is proposed. These methods builds upon the methods outlined in \cite{4812421, Elbassuoni:2011:KSO:2063576.2063615}. The most substantial difference is the use of R-trees to index the spatial dimension of the graph. This is done so that a subgraph containing the keywords can be retrieved given a location. This method starts by finding the closest spatial vertex and uses that as the root for the subgraph. Subgraphs rooted in a spatial node requires different ranking, and there is proposed a set of rules in the paper.\\
R-trees are designed around the spatial dimension, but some work on modifications to include a temporal dimension has been done. Some research done on spatio-temporal tree structures include \cite{Tao:2003:TOS:1315451.1315519, r-tree-spatio-temporal}. Most of the spatio-temporal trees are however created to be able to query and index moving or frequently updated objects. To include a temporal dimension some differences in the tree structures are made. Most approaches builds on the R-tree, but modifies the tree in different aspects. For the purpose of this thesis a different tree structure is not needed.\\ % look into temporal retrieval as facts in graph/ inverted index

% TPR-trees \cite{} uses some tech ... This makes them most suited to index moving objects, with as spatio-temporal data to indicate where an object is at a given time. The work done in \cite{} builds heavily on the R-tree, and makes it suited for ...

\clearpage